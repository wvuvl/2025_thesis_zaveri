% Chapter Template

\chapter{Conclusions and Future Work} % Main chapter title

\label{Chapter6} % Change X to a consecutive number; for referencing this chapter elsewhere, use \ref{ChapterX}


\section{Conclusions}
We introduce SiamABC, a new Siamese visual tracker that improves the trade-off between the computational requirements and the OOD generalization ability, thus expanding the horizon of applicability of visual trackers in-the-wild under resource constraints. We have shown that it can be as fast as FEAR-XS, while being significantly more accurate with an evaluation over 11 benchmarks. We have also shown the superior ability of SiamABC in OOD generalization by reaching near-SOTA accuracies on the challenging OOD benchmark AVisT, with a significant improvement over the efficient Transformer-based SOTA methods. We credit this achievement to the four major technical contributions of the approach that include the use of a dual-search-region, the fast filtration layer FMT, the TRL loss, and the introduction, for the first time, of the dynamic TTA during tracking. 


\section{Future Work}
The results on various benchmarks suggest that our approach is indeed time-efficient and performing reasonably well; however, it is far from a perfect tracker. For accurate tracking, template-matching algorithms still fall short, and our approach is still, at the core, a template-matching algorithm. In order to improve the tracker even further, we need to involve the object's inertia over a period of time. Our tracker is not able to handle it at the moment. All in all, there are still many holes to fill and much more to uncover. We hope our tracker will further the research in terms of tracking speed and efficient use of information fusion even incorporating inertia. Furthermore, since efficient tracking will come at the price of model reduction and, with that, a trade-off with accuracy, we want to encourage efficient tracking in terms of adaptation at execution as it is a possible solution that could be used in the wild.

